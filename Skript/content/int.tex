
\section{Mehrdimensionale Integralrechnung}

\subsection{Lebesgue Maß}
Für Intervalle $I_i : = (a_i,b_i) \subset \mathbb{R}$ nennen wir $I := I_1 \times \cdots \times I_n$ einen Quader und definieren sein Volumen
\begin{align*}
\text{vol}_n (I):=   \prod_{i = 1}^n (b_i -a_i)  
\end{align*}
Mit $I(n): = \{   I_1 \times \cdots \times I_n  \; | \;  I_i := (a_i, b_i) \subset \mathbb{R} \}$ bezeichnen wir die Menge aller $n$-dimensionalen Quader. 
\begin{Definition}
Für eine Menge $A \subset \mathbb{R}^n$ definieren wir das Lebuesgsche äußere Maß
\end{Definition}
\subsection{Lebesgue Integral}

\subsubsection*{Anwendung: Fourierreihen, Fouriertransformation und FFT} 